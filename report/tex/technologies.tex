\chapter{Technologies}
In this chapter the technology choices for the exploration process are explained and brief introductions are given to these technologies.

It starts with the Haskell programming language, and then goes into the SCTP network protocol.

\section{The functional language}
\paragraph{Why Haskell}
The programming language we are focussing on is Haskell.
This language is of special interest because of several properties it has that makes it suitable for implementing a network protocol.
First of all Haskell aims to be an as pure program language as possible. This makes it harder to achieve side effects which are important to a network protocol, but it does simplify algorithms and makes it easier to verify correctness when comparing the implementation to a specification.
In the currently most popular implementation of Haskell side effects are introduced by executing them in a so called Monad.
How this works will be explained in a later chapter.
% When code is free of side effects the amount of bugs that can occur in that code is significantly reduced.
The absence of side effects in pure Haskell makes it easier to understand complex behaviour and even prove that code follows contracts.
This aids greatly when implementing network protocols which usually are well specified and follow strict contracts.
That Haskell features memory management and immutable data also improves the security of the implementation by moving the responsibility of protecting from buffer overruns and other unintended state manipulation to the runtime system, which has been thorougly battle tested.

Other functional programming languages also have interesting properties.
Erlang for example claims to be a language with fault tolerance and concurrency as a prime feature\cite{armstrong_concurrent_1993}.
The Scala language claims to be scalable by being easily composable\cite{odersky_overview_????}.
These and other features have made them popular languages amongst network oriented applications. % cite?
These languages do not have a clear separation between pure and non-pure code so were less interesting for this report.
\paragraph{A short introduction to Haskell}
A simple example that shows off some of Haskell's basic features is shown below:

\begin{lstlisting}[caption={Fibonacci}]
-- Returns the n-th number in the Fibonacci sequence
fib :: Int -> Int
fib 0 = 1
fib 1 = 1
fib n = a + b
  where
    a = fib (n-1)
    b = fib (n-2)
\end{lstlisting}

The first line in this code snippet is a comment since it is prefixed with \lstinline{--}. Haskell also allows block comments beginning with \{- and ending with -\}.

The second line is the type signature of the fib function. This is optional but recommended, if it is omitted the types are inferred from the function body. In this case the function is explicitly defined as taking a single machine integer as a parameter, and returning a single integer as the result. 

The function is defined in three clauses. It uses pattern matching to choose between the three. 
The third and fourth lines are the first two clauses of the fib function. The parameter is restricted to the values 0 and 1 respectively, and their implementations are what follows after the '=': to return 1 as a base case of the function.

The fifth line is the general case of the function. Here the parameter is unrestricted and given the name n. The implementation adds two values a and b, they are defined in the where block below the where keyword.

Lines 7 and 8 define the a and b values to be the result of recursive calls to the fib function with the parameter diminished by one and two respectively. They are executed in the scope of the where clause. This is denoted by indenting the lines deeper than the where keyword is indented.

Below is an example of a function that takes multiple parameters, one of which is a function itself:

\begin{lstlisting}[caption={The map function}]
map :: (a -> b) -> [a] -> [b]
map f [] = []
map f (x : xs) = f x : (map f xs)
\end{lstlisting}

The first parameter of the map function has the type \lstinline{(a->b)}, this means it takes a function that takes a value of type a and returns a value of type b. 
The lowercase types mean that they could be replaced by any type, so the function is generic in a and b.

The map function takes a second argument, a list of values of type a, and returns a list of values of type b. 

The implementation of this map function is recursive and like the fib function uses pattern matching to catch the base case, when the list is empty, it just returns the empty list.

In the general case pattern matching is used in an interesting manner. The list parameter is destructured. In Haskell the colon sign is a list constructor, on its left side it has the first element of the list, on the right side a list with the rest of the elements. The pattern match makes use of this to give the first element and the rest of the elements separate names, x and xs.

The function body itself is simple, the function parameter f is applied to the first element, and a list is constructed from it using the colon with on the right side the map function applied to the rest of the elements.

One might wonder why the parameters are separated by arrows as were they return values too, this is because they actually are return values. This is illustrated in the next snippet. 

\begin{lstlisting}[caption={The map function}]
add :: Int -> [Int] -> [Int]
add a = map (+a)
\end{lstlisting}

This function add takes a list of integers and adds an integer to each of them. It does this by returning the map function with only its first parameter supplied, the function parameter. This concept is called a curry, the map function with one parameter supplied is a function of the following type: \lstinline{[a] -> [b]}, a function that takes just the last parameter and returns the mapped over elements, in this case the types of a and b are restricted to Int by the add function.


Below is an example of an algebraic type defined in Haskell:

\begin{lstlisting}[caption={IP address type}]
data IpAddress = IPv4 Word32 |
                 IPv6 (Word32, Word32, Word32, Word32)
\end{lstlisting}

The keyword 'data' denotes the definition of an algebraic data type. Here 'IpAddress' is the name of the type we're defining.
After the '=' operator follow the constructors of the type. An IpAddress here can either be constructed by 'IPv4' or by 'IPv6' which are separated by a '|'.
After the constructor follows its parameters, in the case of IPv4 it is a single 32-bit word. In the case of IPv6 it is a tuple of four 32-bit words.

To make accessing data in algebraic types a bit easier Haskell has a syntax sugar called record syntax. With this syntax it is possible to both define an algebraic type and accessors for its members like so:
\begin{lstlisting}[caption={SCTP Common Header}]
-- Every SCTP message has the following header
data CommonHeader = CommonHeader {
  sourcePortNumber :: PortNum,
  destinationPortNumber :: PortNum,
  verificationTag :: VerificationTag,
  checksum :: Word32
}
\end{lstlisting}

This code is expanded like so by the haskell compiler:

\begin{lstlisting}[caption={SCTP Common Header}]
data CommonHeader = CommonHeader PortNum PortNum
                                 VerificationTag Word32
sourcePortNumber :: CommonHeader -> PortNum
sourcePortNumber (CommonHeader p _ _ _) = p
-- .. etc ..
\end{lstlisting}

In this listing the sourcePortNumber takes a CommonHeader as a parameter and uses pattern matching to extract the portnumber. The underscores are don't care wildcards, and a disregarded in the pattern match.


\section{The network protocol}
\paragraph{Why SCTP}
The network protocol we will implement is SCTP.
To test if the I/O styles are useable in practice the chosen network protocol must make heavy use of side effects that interact with eachother.
We look to a transport layer protocol because these protocols deal with all the uncertain properties of transporting data over a network, abstracting them for a large part as a service to the application layer.

A transport layer protocol deals with corrupt data, data that has been received out of order, data that is expected but never arrives, data that is late, data that is congested or congesting, data that is ill-intendedly constructed, data that is to be multiplexed or demultiplexed.

SCTP deals with all of these things, to provide a more reliable data stream for the application layer.
While SCTP deals with all of the same things as TCP does, and even a few more, its specification is quite a bit simpler and more modern, which is why it was chosen for this report.
\paragraph{A short introduction to SCTP}
From the RFC4960\cite{_rfc_????}
\begin{quotation}
SCTP is a reliable transport protocol operating on top of a
   connectionless packet network such as IP.  It offers the following
   services to its users:
\begin{itemize}

   \item acknowledged error-free non-duplicated transfer of user data

   \item data fragmentation to conform to discovered path maximum transmission unit size

   \item sequenced delivery of user messages within multiple streams, with
       an option for order-of-arrival delivery of individual user
       messages

   \item optional bundling of multiple user messages into a single SCTP
       packet

   \item network-level fault tolerance through supporting of multi-homing
       at either or both ends of an association.
\end{itemize}
\end{quotation}
Here an association is an established connection between peers over which data can be transferred. Multi-homing is the feature that allows an association to have more than one address at either end of the association. This makes the association tolerant to network failures where one or more of the peers fail, as long as there is at least one functional peer on either end of the connection.

To set up an SCTP connection a handshake is performed. The connecting party sends a connection request to the listening party. The listening party creates a cookie with information about the connecting party and encrypts it with a secret key and sends it to the connecting party. The connecting party then echoes this cookie. The listening party decrypts the cookie, discovers that the connecting party really is committed to making the connection and then commits itself to the connection. This handshake makes sure the listening party does not commit resources to a connecting party before it is sure the connecting party has not spoofed its address.

After the connection is set up the parties send eachother messages. These messages consist of a common header, containing the source and destination ports and a checksum, and one or more chunks. The chunks can be protocol messages like heartbeat, shutdown or data acknowledgement or they could be data messages as supplied by the application layer.

The data messages are selectively acknowledged, this means an acknowledgement chunk can acknowledge multiple data fragments, even when they were received out of order or with gaps between them. A timer is kept for each sent data fragment when a fragment goes without acknowledgement for too long it is considered lost and will be retransmitted. 
How long a system must wait before retransmitting a fragment is determined by a function that performs statistic analysis on round trip time measurements to make an estimate of the ideal retransmission time. 
