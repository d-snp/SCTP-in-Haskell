\chapter{Process}
\section{Approach}
To explore different styles we have implemented the foundations of the SCTP protocol\cite{_rfc_????} in Haskell. The implementation yields valuable insight into which I/O functions are used and how they interact with eachother.
\section{Monadic Style}
\paragraph{Architecture}
The architecture of the monadic style implementation makes use of events.

The interfacing program first starts the stack. The stack opens an operating system socket and spawns a new thread that performs the blocking read operation on this socket. Every time a packet arrives at the stack the read yields. The stack then goes through a list of registered sockets to see if any socket is open to data on the port the packet is destined for.

The interfacing application can register a socket at the stack by either calling the 'connect' or 'listen' function with an address and an event handler function as parameters.
When the stack finds a socket that accepts data on the right port it call 'socketAcceptMessage' with the message and the source as parameters.

The 'socketAcceptMessage' function checks and decodes the message calls the handle function for the particular type of the message. The handler functions perform actions like creating associations, transferring data and tearing associations down. When an action is performed that the interfacing application could be interested in, such as an incoming connection or received data, the event handler the application registered is called with an event object with relevant information.

Various actions in the protocol have associated timers. For example when a connection is initiated the connecting side starts a timer. If an acknowledgement of the connection attempt is not received before the timer runs out the attempt is retransmitted. The timer runs in a separate thread which repeatedly sleeps a few milliseconds, wakes up and checks wether enough time has passed. If it has the function associated to the timer is called.

The implementation of the architecture has a clear separation between pure and I/O functions. All I/O functions are in the 'socket.hs' file, which is the main interface. All non-I/O functions have been extracted into utility functions and placed into different files. This clear separation allows us to clearly see what I/O actions are performed by each function.

\begin{itemize}
\item To receive and send data, an operating system socket is created and the sendTo and recvFrom functions are called. Whenever a new socket is created a portnumber is also requested from the operating system.
\item To allow for multiple sockets to be created on the stack, a separate thread is spawned and the list of sockets is stored in an MVar.
\item To allow for timed events, network events and application events to affect an association, associations are stored in an MVar and timers are run in separate threads.
\item To create a cookie the getCurrentTime is called. The timers call getCPUTime for small grained timing.
\item To secure associations randomIO is called to generate random numbers.
\item To allow the application interface to have side effects, the event handler is also an IO function.
\end{itemize}

These actions cover the four categories of the awkward squad \cite{jones_tackling_2009}. The threads and MVars have to do with concurrency. The send and receive have to do with input and output. The timers have to do with error detection and recovery. The system calls are interfacing with external libraries and components. As we suspected the network protocol implementation really covers all of the I/O types.

%Separating the I/O from the pure logic was a goal from the start, but was done only when the stack was functioning. Before it was done the project looked a lot like it would in any regular imperative language, assignments and I/O calls were mixed together with protocol logic and 

Another advantage of splitting the I/O and the pure functions is that the pure functions are very easy to test. Haskell even has a quickcheck function that runs random data through the functions to verify properties. This made verifying the message parsing functions and the cookie creation function a breeze.

One I/O implementation specific detail that leaks into the separated files is the concurrent state primitive MVar. The stack, the sockets and the associations all have an MVar member which allow separate threads to mutate their states safely.



% -- why split I/O and pure
% -- why eventhandler?
% -- planning, type driven development
%Hurdles:
% -- eliminating message channels (aka why eventhandler)
% -- Keeping state, eliminating mvars
%Advantages:
% -- End result seems readable, plain to see all I/O, I/O is all the observable behaviour.
%Testing:
% -- Using quickcheck
% -- Testing the I/O
% -- Debugging

% Analysis:
% Mc-Cabe’s cyclomatic number,  Halstead’s  programming  effort, statement  count, and Oviedo’s data flow complexity satisfy our properties
% -- where lies the complexity?
% -- where is the boilerplate, what is hidden?
% -- where are the bugs, missing features?
% -- Haskell performance \cite{epstein_haskell_????}
